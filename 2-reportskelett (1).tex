\documentclass[a4paper,12pt]{article}
\usepackage[swedish]{babel}
\usepackage[utf8]{inputenc}
\usepackage{amsmath, amsthm, amssymb}
% Ändra INTE nästa rad (säger var texten ska typsättas)
\usepackage[a4paper,includeheadfoot,margin=2.54cm]{geometry}
% Ändra INTE nästa rad (som lägger till radnummer till vänster)
\usepackage[left]{lineno}


% Ändra INTE raderna nedan
% Koden är från https://tex.stackexchange.com/questions/43648/
% Den fixar radnumrering av text i närvaro av matematikomgivningar
\newcommand*\patchAmsMathEnvironmentForLineno[1]{%
  \expandafter\let\csname old#1\expandafter\endcsname\csname #1\endcsname
  \expandafter\let\csname oldend#1\expandafter\endcsname\csname end#1\endcsname
  \renewenvironment{#1}%
     {\linenomath\csname old#1\endcsname}%
     {\csname oldend#1\endcsname\endlinenomath}}% 
\newcommand*\patchBothAmsMathEnvironmentsForLineno[1]{%
  \patchAmsMathEnvironmentForLineno{#1}%
  \patchAmsMathEnvironmentForLineno{#1*}}%
\AtBeginDocument{%
\patchBothAmsMathEnvironmentsForLineno{equation}%
\patchBothAmsMathEnvironmentsForLineno{align}%
\patchBothAmsMathEnvironmentsForLineno{flalign}%
\patchBothAmsMathEnvironmentsForLineno{alignat}%
\patchBothAmsMathEnvironmentsForLineno{gather}%
\patchBothAmsMathEnvironmentsForLineno{multline}%
}

% Ändra INTE nästa rad (gör så radnummer skrivs med fet stil)
\renewcommand\linenumberfont{\normalfont\bfseries\small}

\title{Rapportskelett - Byt ut denna titel}
%
\author{Hela ditt fullständiga namn\thanks{email:
        \texttt{xxxxxx-X@student.ltu.se}}\\  
        ~ \\
        Luleå tekniska universitet \\ 
        971 87 Luleå, Sverige}
%          
\date{\today}

\begin{document}

\linenumbers % ger radnumrering

\maketitle

\begin{abstract}
  Här skriver du en kort sammanfattning av rapporten som innehåller
  det viktigaste. 
\end{abstract}

\section{Introduktion}
\label{sec:introduktion}

Här skriver du om helheten, bakgrund, betydelse, ja, en introduktion. 

Sen lägger du till delavsnitt nedan för uppgifter som behandlas.

\section{En (del-) uppgift och dess lösning}
\label{sec:uppg1}

Beskriv uppgiften och var noga med att få med alla förutsättningar.
Kom ihåg att numrera ekvationer som refereras till i löptexten.

\section{Nästa (del-) uppgift}
\label{sec:uppg2}

\section{Och ännu nästa (del-) uppgift...}
\label{sec:uppgN}

\section{Diskussion [och slutsatser]}
\label{sec:disk}

Sammanfatta vad som avhandlats i rapporten, vad du kommit fram till,
och sätt det i sitt sammanhang. 
%
\begin{thebibliography}{99}
%
\bibitem{latexcompanion} 
Michel Goossens, Frank Mittelbach, and Alexander Samarin. 
\textit{The \LaTeX\ Companion}. 
Addison-Wesley, Reading, Massachusetts, 1993.
%
\bibitem{einstein} 
Albert Einstein. 
\textit{Zur Elektrodynamik bewegter K{\"o}rper}. (German) 
[\textit{On the electrodynamics of moving bodies}]. 
Annalen der Physik, 322(10):891–921, 1905.
%
\end{thebibliography}
%
\end{document}
